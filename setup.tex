\documentclass[11pt]{memoir}
\usepackage{amsmath}
\usepackage{amssymb}
\usepackage{amsthm}
\usepackage{fancyhdr}
\usepackage{listings}
\usepackage{booktabs}
\usepackage{xparse}
\NewDocumentCommand{\DIV}{om}{%
  \IfValueT{#1}{\setcounter{#2}{\numexpr#1-1\relax}}%
  \csname #2\endcsname
}


\begin{document}
\DIV[1]{section}{Dependancies}
\begin{center}
\begin{tabular}{@{}|l|l|@{}}
\toprule
Library Name                & Version     \\ \midrule
kombu                       & 4.2.0       \\ \midrule
django                      & 2.0.3       \\ \midrule
google\_auth\_oauthlib      & 0.2.0       \\ \midrule
psycopg2                    & 2.7.4       \\ \midrule
youtube\_dl                 & 2018.3.20   \\ \midrule
google\_api\_python\_client & 1.6.7       \\ \midrule
celery                      & 4.1.1       \\ \midrule
redis                       & 4.0.9       \\ \midrule
protobuf                    & 3.5.2.post1 \\ \bottomrule
\end{tabular}
\end{center}
To start the django project, start the django localhost server with
$$\texttt{python manage.py runserver}$$
In a seperate terminal window, ensure redis is running with
$$\texttt{redis-server}$$
In another terminal window, start a celery worker with
$$\texttt{celery -A crawler worker -l info}$$
To fully terminate the celery worker, run
$$\texttt{pkill -f celery}$$
\end{document}
